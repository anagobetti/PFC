\addcontentsline{toc}{chapter}{Resumo}

\begin{center}
\huge{{\bf Resumo}}
\vspace{2cm}
\end{center}

Esse projeto tem como objetivo desenvolver um sistema que permita a leitura autom�tica de documentos pessoais. Para isso, ser�o utilizadas t�cnicas de vis�o computacional e reconhecimento �tico de Caracteres (OCR). Dessa forma, o sistema desenvolvido tem como objetivo n�o somente digitalizar e ler os documentos, mas identificar as �reas de interesse, nas quais est�o contidos os dados e ele deve ser flex�vel quando ao meio de entrada da imagem do documento e permitir imagens scaneadas com diferentes resolu��es e configura��es de contraste e brilho. Destaca-se a import�ncia deste sistema na automa��o comercial durante o processo de cadastramento de clientes, por exemplo, durante os check-in de hot�is, uma vez que a digitaliza��o e interpreta��o dos dados contidos no documento de identifica��o por um computador � feita de forma muito mais r�pida e � menos propensa a erros do que a leitura feita por um operador, que est� sujeito � diversos fatores que podem influenciar na sua efici�ncia. 

Dessa forma, percebe-se que a exist�ncia de ferramentas que n�o s� convertam arquivos do formato f�sico para o formato digital, mas que tamb�m leiam e interpretem os dados seria de grande utilidade, uma vez que simplificaria o trabalho, reduzindo o tempo de espera em diversas atividades, como nos casos em que � necess�rio realizar um cadastro pessoal, como ao alugar um quarto de hotel ou alugar um carro ou equipamento.

 
\clearpage
\thispagestyle{empty}
\cleardoublepage

